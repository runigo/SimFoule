%
%%%%%%%%%%%%%%%%%%%%%
\section{Le batiment}
%%%%%%%%%%%%%%%%%%%%%
%
Cette section traite de la modélisation d'un étage de batiment
%
\subsection{Cellule}
Un étage du batiment est discrétisé en cellule (de l'ordre de 1 m $\times$ 1 m). Une cellule peut contenir un mur ou être libre aux mobiles. Elle peut également être une sortie de l'étage. Elle contient une information (sens et direction à suivre) permettant de diriger les mobiles vers la sortie la plus proche. Une autre information pourrait être la deuxième sortie la plus proche donnant au mobile un choix à faire.
%
%
\subsection{Étage}
Un étage est l'ensemble des cellules, il consiste en un plan en deux dimensions des murs de l'étage. Un étage peut posséder une ou plusieurs sortie. Les sorties des étages supérieurs sont des escaliers donnant sur les entrées de l'étage inférieur. Un algorithme permet de déterminer le plus court chemin pour accéder à la sortie la plus proche et initialise cette orientation dans les cellules.
%
%
\subsection{Batiment}
Un batiment est composé de plusieurs étages reliés par des escaliers. Il dispose d'un étage 0 pouvant posséder une ou plusieurs sorties. Les sorties des étages supérieurs ammènent à des escaliers donnant accès à l'étage inférieur.
%%%%%%%%%%%%%%%%%%%%%%%%%%%%%%%%%%%%%%%%%%%%%%%%%%%%%%%%%%%%%%%%%%%%%%%%%%%%%%%%%%%%%
