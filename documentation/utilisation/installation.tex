
%%%%%%%%%%%%%%%%%%%%%%%%%%%%%%%%%%
%
%%%%%%%%%%%%%%%%%%%%%%%%%%%%%%%%%%

\section{Présentation du simulateur}
%
Le simulateur SimFoule est un programme informatique écrit en C et qui utilise la librairie SDL2. Il donne une représentation graphique d'une foule se déplaçant dans un batiment. Une évacuation du batiment peut être déclenchée.
%
%
\section{Installation du simulateur}
Cette section traite de l'installation du simulateur SimFoule sur un système d'exploitation de type debian. Le téléchargement se fait avec un navigateur internet, la compilation et l'exécution se font dans un terminal. L'installation des outils de compilation nécessite les privilèges du super-utilisateur.
\begin{itemize}[leftmargin=1cm, label=\ding{32}, itemsep=0pt]%\end{itemize}
\item {\bf Installation des outils de compilation}
	\begin{itemize}[leftmargin=1cm, label=\ding{32}, itemsep=0pt]
	\item \texttt{sudo apt-get install gcc make libsdl2-dev}
	\end{itemize}
\item {\bf Téléchargement des sources}
	\begin{itemize}[leftmargin=1cm, label=\ding{32}, itemsep=0pt]
	\item Télécharger le fichier \texttt{.zip} sur github
		\begin{itemize}[leftmargin=1cm, label=\ding{32}, itemsep=0pt]
		\item \texttt{https://github.com/runigo/SimFoule/archive/master.zip}
		\end{itemize}
	\item Décompresser le fichier \texttt{.zip}
		\begin{itemize}[leftmargin=1cm, label=\ding{32}, itemsep=0pt]
		\item \texttt{unzip SimFoule-master.zip}
		\end{itemize}
	\end{itemize}
\item {\bf Compilation}
	\begin{itemize}[leftmargin=1cm, label=\ding{32}, itemsep=0pt]
	\item La commande \texttt{make} dans le répertoire des sources produit un fichier exécutable :
		\begin{itemize}[leftmargin=1cm, label=\ding{32}, itemsep=0pt]
		\item \texttt{SimFoule}
		\end{itemize}
	\end{itemize}

\item {\bf Exécution}
	\begin{itemize}[leftmargin=1cm, label=\ding{32}, itemsep=0pt]
	\item En ligne de commande, avec d'éventuelles options
		\begin{itemize}[leftmargin=1cm, label=\ding{32}, itemsep=0pt]
		\item \texttt{./SimFoule [OPTION]}
		\end{itemize}
	\end{itemize}

Lancé dans le répertoire des sources, afin de bénéficier du répertoire \texttt{donnees/enregistrement/} contenant les fichiers d'initialisations et donnant accès aux fonctions de sauvegardes.


La fenêtre graphique donne une représentation de la simulation, le terminal affiche les informations.
\end{itemize}




