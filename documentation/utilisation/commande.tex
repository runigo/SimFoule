%\chapter{Utilisation des simulateurs}
%%%%%%%%%%%%%%%%%%%%%%%%%%%%%%%%%%%%%
\section{Commande du simulateur}
%%%%%%%%%%%%%%%%%%%%%%%%%%%%%%%%%%%%%
%
Cette section traite des interactions entre le programme et l'utilisateur.
%
\subsection{Résumé des options}
\begin{center}
\begin{tabular}{ccccc}
option & & valeur &  & commande \\
%\hline
\texttt{fond} &  & 0 < \texttt{fond} < 255 &  & Couleur du fond \\
\texttt{dt} &  & \texttt{DT\_MIN} < \texttt{dt} < \texttt{DT\_MAX} &  & discrétisation du temps \\
\texttt{duree} &  & 0 < \texttt{duree} < \texttt{DUREE\_MAX} &  & Nombre d'évolution du \\
 &  &  &  & système entre deux affichages \\
\texttt{mode} &  & \texttt{mode} = + ou - 1 &  & Mode -1 : Wait, 1 : Poll \\
\texttt{pause} &  & 5 < \texttt{pause} < 555 &  & temps de pause en ms \\
 &  &  &  & entre les affichages \\
%{\sf F9}, {\sf F10}, {\sf F11}, {\sf F12}{\sf Entrée}
\texttt{initial} &  & \texttt{INITIAL\_MIN} < \texttt{initial} < \texttt{INITIAL\_MAX} &  & Numéro du fichier d'initialisation \\
\texttt{masse} &  & \texttt{MASSE\_MIN} < \texttt{masse} < \texttt{MASSE\_MAX} &  & Masse des mobiles \\
\texttt{nervosite} &  & \texttt{NERVOSITE\_MIN} < \texttt{nervosite} < \texttt{NERVOSITE\_MAX} &  & Nervosité des mobiles \\%{\sf e}, {\sf d}
%\texttt{nombre} & (nombre > 0 \& nombre < 1099) &  & Nombre de mobile\\
\texttt{dessineAngle} &  & = \texttt{0} ou \texttt{1} &  & Dessine ou non les directions \\
\texttt{dessineMur} &  & = \texttt{0} ou \texttt{1} &  & Dessine ou non les murs \\
\texttt{dessineMobile} &  & = \texttt{0} ou \texttt{1} &  & Dessine ou non les mobiles \\

\\
\texttt{aide} &  &  &  & Affiche l'aide \\
\texttt{help} &  &  &  & Affiche l'aide \\
\end{tabular}
\end{center}
%
\subsection{Résumé du clavier}
\begin{center}
\begin{tabular}{ccccc}
option & & clavier &  & commande \\
%\hline
%\texttt{fond} &  & \texttt{fond}>0 \& \texttt{fond}<255 &  & Couleur du fond \\
%\texttt{dt} &  & \texttt{dt} > 0.0 \& \texttt{dt} < \texttt{DT\_MAX} &  & discrétisation du temps \\
 &  & {\sf F1}, {\sf F5} &  & Informmation \\

\texttt{duree} &  & {\sf F9}, {\sf F10}, {\sf F11}, {\sf F12} &  & Nombre d'évolution du \\
 &  &  &  & système entre deux affichages \\
\texttt{mode} &  & {\sf Entrée} &  & Avec ou sans attente \\
%\texttt{pause} &  & 5 < \texttt{pause} < 555 &  & temps de pause en ms \\
% &  &  &  & entre les affichages \\
%
\texttt{initial} &  & \texttt{Maj} \texttt{a}..\texttt{z} &  & Réinitialisation \\
 &  & \texttt{Ctrl} \texttt{a}..\texttt{z} &  & Sauvegarde \\
 &  & \texttt{Ctrl Maj} \texttt{a}..\texttt{z} &  & Réinitialisation implicite \\
%\texttt{masse} &  & a q &  & Masse des mobiles \\
%\texttt{nervosite} &  & z s &  & Nervosité des mobiles \\%{\sf e}, {\sf d}
%\texttt{nombre} & (nombre > 0 \& nombre < 1099) &  & Nombre de mobile\\
\texttt{dessineAngle} &  & \texttt{F6} &  & Dessine ou non les directions \\
\texttt{dessineMur} &  & \texttt{F7} &  & Dessine ou non les murs \\
\texttt{dessineMobile} &  & \texttt{F8} &  & Dessine ou non les mobiles \\

 &  & \texttt{Échap} &  & Mode construction, simulation \\

 &  & \texttt{z, s} &  & Augmente, diminue la nervosité des mobiles \\
 &  & \texttt{e, d} &  & Augmente, diminue la célérité des mobiles \\

 &  & \texttt{a, q} &  & Augmente, diminue la masse des mobiles \\
 &  & \texttt{z, s} &  & Augmente, diminue la nervosité des mobiles \\
 &  & \texttt{e, d} &  & Augmente, diminue la célérité des mobiles \\

 &  & \texttt{w} &  & Boucle les configurations enregistées  \\
 &  & \texttt{x} &  & Répétition de la configuration \\
 &  & \texttt{c} &  & Configuration implicite \\

 &  & \texttt{u} &  & Entrée  \\
 &  & \texttt{i} &  & Sortie \\
 &  & \texttt{o} &  & Vide \\
 &  & \texttt{p} &  & Mobile \\
 &  & \texttt{m} &  & Mur \\
\\
%\texttt{aide} &  & () &  & Affiche l'aide \\
%\texttt{help} &  & () &  & Affiche l'aide \\
\end{tabular}
\end{center}
%
\subsection{Fichier d'initialisation}
%
Les fichiers d'initialisations se trouvent dans le répertoire \texttt{donnees/enregistrement/}. Ce répertoire et ces fichiers sont nécessaires pour activer les possibilité de réinitialisation.
%
\subsubsection{Éditer les fichiers d'initialisations}
Les fichiers d'initialisations peuvent être édités afin de créer des situations. Ils contiennent des entiers séparés par des espaces.

La première ligne indique les dimensions spatiales du batiment, les lignes suivantes correspondantent au statut des cellules.

Les entiers reconnus lors de l'initialisation des cellules sont

\begin{center}
\begin{tabular}{cccccc}
entiers & 0 & 1 & 2 &  & 9 \\
%\hline
statut & vide & mur & sortie &  & mobile \\
\end{tabular}
\end{center}
%
\subsubsection{Réinitialisation}
La combinaison des touches \texttt{Maj [a..z]} réinitialise le système dans la configuration correspondante au fichier, eventuellement les configurations implicites si aucune modification n'a été effectué.
%
\subsection{Dessiner des conditions initiales}
La touche \texttt{Échap} donne accés au mode initialisation permettant la construction de conditions initiales.

La sauvegarde est alors possible grâce à la combinaison des touches \texttt{Ctrl [a..z]}. La réinitialisation se fait par la combinaison des touches \texttt{Maj [a..z]}. La combinaison des touches \texttt{Ctrl Maj [a..z]} permet la réinitialisation dans les configurations implicites.
En mode "conditions initiale" la souris permet de dessiner des conditions initiales. Une configuration ainsi dessinée peut alors être enregistrée dans les fichiers \texttt{situation\_q.simfoule} à \texttt{situation\_h.simfoule} avec les touches \texttt{Maj-w} à \texttt{Maj-n}. elle sera alors accessible à l'aide des touches \texttt{Maj-q} à \texttt{Maj-h}.











%%%%%%%%%%%%%%%%%%%%%%%%%%%%%%%%%%%%%%%%%%%%%%%%%
