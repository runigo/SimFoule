%\chapter{Utilisation des simulateurs}
%%%%%%%%%%%%%%%%%%%%%%%%%%%%%%%%%%%%%
\section{Commande du simulateur}
%%%%%%%%%%%%%%%%%%%%%%%%%%%%%%%%%%%%%

Cette section traite des interactions entre le programme et l'utilisateur.

\subsection{Résumé des options}
\begin{center}
\begin{tabular}{cccc}
option & valeur & clavier & commande \\
%\hline
{\texttt fond} & (fond>0 \& fond<255) &  & Couleur du fond \\
{\texttt soliton} & (soliton > -99 \& soliton < 99) & {\sf y},{\sf h} & déphasage entre les extrémitées *\\
{\texttt dt} & (dt > 0.0 \& dt < DT\_MAX) &  & discrétisation du temps \\
{\texttt frequence} & () & {\sf p}, {\sf m} & fréquence du générateur \\
{\texttt dissipation} & () & {\sf e}, {\sf d} & dissipation \\
{\texttt equation} & (equation > 0 \& equation < 5) & {\sf F1}, {\sf F2}, {\sf F3}, {\sf F4} & choix de l'équation \\
{\texttt pause} & (pause > 5 || pause < 555) &  & temps de pause en ms \\
{\texttt duree} & () & {\sf F11}, {\sf F12} & Nombre d'évolution du système entre les affichages \\
{\texttt mode} & () & {\sf Entrée} & Mode -1 : Wait, 1 : Poll \\
{\texttt nombre} & (nombre > 0 \& nombre < 1099) &  & Nombre de pendules **\\
{\texttt aide} & () &  & Affiche l'aide \\
{\texttt help} & () &  & Affiche l'aide \\
\end{tabular}
\end{center}

