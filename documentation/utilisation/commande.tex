%\chapter{Utilisation des simulateurs}
%%%%%%%%%%%%%%%%%%%%%%%%%%%%%%%%%%%%%
\section{Commande du simulateur}
%%%%%%%%%%%%%%%%%%%%%%%%%%%%%%%%%%%%%

Cette section traite des interactions entre le programme et l'utilisateur.

\subsection{Résumé des options}
\begin{center}
\begin{tabular}{cccc}
option & valeur & clavier & commande \\
%\hline
{\texttt fond} & (fond>0 \& fond<255) &  & Couleur du fond \\
{\texttt dt} & (dt > 0.0 \& dt < DT\_MAX) &  & discrétisation du temps \\
{\texttt nervosite} & 0.0 < nervosite < NERVOSITE\_MAX & {\sf e}, {\sf d} & Nervosité des humains \\
{\texttt pause} & (pause > 5 || pause < 555) &  & temps de pause en ms \\
{\texttt duree} & () & {\sf F11}, {\sf F12} & Nombre d'évolution du \\
 &  &  & système entre les affichages \\
{\texttt mode} & () & {\sf Entrée} & Mode -1 : Wait, 1 : Poll \\
{\texttt nombre} & (nombre > 0 \& nombre < 1099) &  & Nombre d'humain\\
{\texttt aide} & () &  & Affiche l'aide \\
{\texttt help} & () &  & Affiche l'aide \\
\end{tabular}
\end{center}

\subsection{Fichier d'initialisation}
Les fichiers d'initialisation sont dans le répertoire donnees/enregistrement/. Ils peuvent être éditer afin de créer des situations. Ils doivent être adaptés aux constantes définies dans le fichiers donnees/constantes.
