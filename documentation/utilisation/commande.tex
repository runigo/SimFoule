%\chapter{Utilisation des simulateurs}
%%%%%%%%%%%%%%%%%%%%%%%%%%%%%%%%%%%%%
\section{Commande du simulateur}
%%%%%%%%%%%%%%%%%%%%%%%%%%%%%%%%%%%%%
%
Cette section traite des interactions entre le programme et l'utilisateur.
%
\subsection{Résumé des options}
\begin{center}
\begin{tabular}{ccccc}
option & & valeur &  & commande \\
%\hline
\texttt{fond} &  & 0 < \texttt{fond} < 255 &  & Couleur du fond \\
\texttt{dt} &  & \texttt{DT\_MIN} < \texttt{dt} < \texttt{DT\_MAX} &  & discrétisation du temps \\
\texttt{duree} &  & 0 < \texttt{duree} < \texttt{DUREE\_MAX} &  & Nombre d'évolution du \\
 &  &  &  & système entre deux affichages \\
\texttt{mode} &  & \texttt{mode} = + ou - 1 &  & Mode -1 : Wait, 1 : Poll \\
\texttt{pause} &  & 5 < \texttt{pause} < 555 &  & temps de pause en ms \\
 &  &  &  & entre les affichages \\
%{\sf F9}, {\sf F10}, {\sf F11}, {\sf F12}{\sf Entrée}
\texttt{initial} &  & \texttt{INITIAL\_MIN} < \texttt{initial} < \texttt{INITIAL\_MAX} &  & Numéro du fichier d'initialisation \\
\texttt{masse} &  & \texttt{MASSE\_MIN} < \texttt{masse} < \texttt{MASSE\_MAX} &  & Masse des mobiles \\
\texttt{nervosite} &  & \texttt{NERVOSITE\_MIN} < \texttt{nervosite} < \texttt{NERVOSITE\_MAX} &  & Nervosité des mobiles \\%{\sf e}, {\sf d}
%\texttt{nombre} & (nombre > 0 \& nombre < 1099) &  & Nombre de mobile\\
\texttt{dessineAngle} &  & = \texttt{0} ou \texttt{1} &  & Dessine ou non les directions \\
\texttt{dessineMur} &  & = \texttt{0} ou \texttt{1} &  & Dessine ou non les murs \\
\texttt{dessineMobile} &  & = \texttt{0} ou \texttt{1} &  & Dessine ou non les mobiles \\

\\
\texttt{aide} &  &  &  & Affiche l'aide \\
\texttt{help} &  &  &  & Affiche l'aide \\
\end{tabular}
\end{center}
%
\subsection{Résumé du clavier}
\begin{center}
\begin{tabular}{ccccc}
option & & clavier &  & commande \\
%\hline
%\texttt{fond} &  & \texttt{fond}>0 \& \texttt{fond}<255 &  & Couleur du fond \\
%\texttt{dt} &  & \texttt{dt} > 0.0 \& \texttt{dt} < \texttt{DT\_MAX} &  & discrétisation du temps \\
\texttt{duree} &  & 0 < {\sf F9}, {\sf F10}, {\sf F11}, {\sf F12} &  & Nombre d'évolution du \\
 &  &  &  & système entre deux affichages \\
\texttt{mode} &  & {\sf Entrée} &  & Avec ou sans attente \\
%\texttt{pause} &  & 5 < \texttt{pause} < 555 &  & temps de pause en ms \\
% &  &  &  & entre les affichages \\
%
\texttt{initial} &  & \texttt{Maj} \texttt{a}..\texttt{t} et \texttt{q}..\texttt{g} &  & Réinitialisation \\
%\texttt{masse} &  & a q &  & Masse des mobiles \\
%\texttt{nervosite} &  & z s &  & Nervosité des mobiles \\%{\sf e}, {\sf d}
%\texttt{nombre} & (nombre > 0 \& nombre < 1099) &  & Nombre de mobile\\
\texttt{dessineAngle} &  & \texttt{F6} &  & Dessine ou non les directions \\
\texttt{dessineMur} &  & \texttt{F7} &  & Dessine ou non les murs \\
\texttt{dessineMobile} &  & \texttt{F8} &  & Dessine ou non les mobiles \\

\\
%\texttt{aide} &  & () &  & Affiche l'aide \\
%\texttt{help} &  & () &  & Affiche l'aide \\
\end{tabular}
\end{center}
%
\subsection{Fichier d'initialisation}
%
Les fichiers d'initialisations se trouvent dans le répertoire \texttt{donnees/enregistrement/}. Ce répertoire et ces fichiers sont nécessaires pour activer les possibilité de réinitialisation.
%
\subsubsection{Éditer les fichiers d'initialisations}
Les fichiers d'initialisations peuvent être édités afin de créer des situations. Ils doivent être adaptés aux constantes \texttt{BATIMENT\_X} et \texttt{BATIMENT\_Y} définies dans le fichiers \texttt{donnees/constantes.h}. Il contiennent des entiers séparés par des espaces correspondant au statut des cellules. Une ligne doit contenir \texttt{BATIMENT\_X} entiers, et le fichier doit contenir \texttt{BATIMENT\_Y} lignes. Les entiers reconnus lors de l'initialisation sont

\begin{center}
\begin{tabular}{cccccc}
entiers & 0 & 1 & 2 &  & 9 \\
%\hline
statut & vide & mur & sortie &  & mobile \\
\end{tabular}
\end{center}
%
%\subsubsection{Dessiner des conditions initiales}
%En mode "conditions initiale" la souris permet de dessiner des conditions initiales. Une configuration ainsi dessinée peut alors être enregistrée dans les fichiers \texttt{situation\_q.simfoule} à \texttt{situation\_h.simfoule} avec les touches \texttt{Maj-w} à \texttt{Maj-n}. elle sera alors accessible à l'aide des touches \texttt{Maj-q} à \texttt{Maj-h}.











%%%%%%%%%%%%%%%%%%%%%%%%%%%%%%%%%%%%%%%%%%%%%%%%%
