%
%%%%%%%%%%%%%%%%%%%%%%%%%%%%
\section{Vision des mobiles}
%%%%%%%%%%%%%%%%%%%%%%%%%%%%
%
Dans la version 1.3, les choix de directions ne se font que dans les cellules ayant été initialisées suivant une direction "diagonale". La direction souhaitée est alors choisies suivant le nombre de mobiles présents dans les cellules à atteindre.
La version 1.4 doit mettre en place :
%\begin{itemize}[label=\ding{32}, leftmargin=1.1cm]
\begin{itemize}[leftmargin=2cm]
\item Un choix pour les cellules dont la direction initialisée n'est pas "diagonale".
\item Une diminution du module de la vitesse souhaitée si les cellules à ateindre sont occupées.
\end{itemize}
%
La version 1.3.6 prend en compte le taux d'occupation des cellules à ateindre afin de faire un choix.
%
\subsection{Modélisation}
%
Une cellule contient les informations suivantes :
\begin{itemize}[leftmargin=2cm]
\item int statut;		// 0:libre, 1:mur, 2:sortie, 3:entrée, 9:mobile lors de l'initialisation
\item int visite;		// 1 si initialisé, -1 si visite en cours
\item int distance;		// Distance à la sortie
\item int issue;		// Nombre de voisines "direct" à ateindre
\item float interet[8];		// Intérêt à aller dans la direction
\item float note[8];		// Intérêt - nombre(cellule voisine)
\item int sens;			// Meilleur sens
\item int nombre;		// nombre de mobile.
\end{itemize}
%
\subsection{Champ de vision}
%
\begin{center}
	\begin{tabular}{rcl}
	 & \\
	\end{tabular}
\end{center}
%
%
%
\subsubsection{Limitation des valeurs des variables}
%

\begin{itemize}[label=\ding{32}, leftmargin=2cm]
\item 
\end{itemize}
%
{\texttt donnees/constantes.c}
%
