%
%%%%%%%%%%%%%%%%%%%%%
\section{Valeurs implicites}
%%%%%%%%%%%%%%%%%%%%%
%
%
\subsection{Réglage de dt, durée et pause}
Une incrémentation du système correspond à une avancée dans le temps de \texttt{dt}.
Empiriquement, un affichage graphique par 30 ms, est obtenue avec une pause de l'ordre de 25 ms. Si à chaque affichage correspond à une dizaine d'incrémentation de dt, 
\begin{center}
	\begin{tabular}{rcl}
	dt $\times$ duree & = & delay\\
	dt $\times$ 10 & = & 0,03\\
	dt & = & 0,003\\
	\end{tabular}
\end{center}
La valeur de {\texttt duree} peut être changée dynamiquement avec les touches {\texttt F11} et {\texttt F12} afin de faire varier la vitese de la simulation. La valeur de dt peut être réglée avec l'option {\texttt dt} au démarrage du programme, celle de delay par l'option {\texttt pause}.

Dans SimFoule, les valeurs implicites de {\texttt dt} et {\texttt duree} sont égale à 0,003 et 11, celle de pause est égale à 25. Ces valeurs peuvent être affinée suivant les objectifs et les capacités de l'ordinateur.
%
%
%
%
%
\subsubsection{Limitation des valeurs des variables}
%
Au delà de certaines valeurs de certain paramètres dynamiques, la simulation s'éloigne du comportement physique.
\begin{itemize}[label=\ding{32}, leftmargin=2cm]
\item Pour des raisons de discrétisation
\item En raisons de possibles erreurs d'algorithme
\item En raisons de possibles erreurs d'écriture
\end{itemize}
%
%
\subsubsection{Paramètres physiques}
%
%
\subsubsection{Paramètres dynamiques}
%
Au delà de certaines valeurs de certain paramètres dynamiques, la simulation s'éloigne d'une évolution physique réaliste.
\begin{itemize}[label=\ding{32}, leftmargin=2cm]
\item Vitesse des mobiles
\item Distance entre les mobiles
\item Énergie des mobiles
\end{itemize}
%
