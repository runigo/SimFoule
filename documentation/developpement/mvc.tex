%
%%%%%%%%%%%%%%%%%%%%%
\section{Modèle Vue Controleur}
%%%%%%%%%%%%%%%%%%%%%
%
%
\subsection{Les répertoires du simulateur}
\begin{itemize}[leftmargin=2cm]
\item \texttt{donnees} : Inclusion des librairies standard, définitions des constantes et des valeurs initiales de la simulation.
%\item \texttt{fonctions} : Outils mathématique. Fonctions et projection du système
\item \texttt{modele} : Système simulé. 
\item \texttt{graphisme} : Représentation graphique et affichage. Utilisation de la librairie SDL2
\item \texttt{controle} : Liaison entre le système et l'interface graphique 
\item \texttt{objet} : Nécessaire à la compilation
\item \texttt{documentation} : Documentation du simulateur et bibliographie.
\end{itemize}
%
\subsection{Le modèle}
Le système est constitué d'une foule et d'un batiment. La foule est constituée d'humains, le batiment est constitué d'étage et d'escalier, eux même constitués de cellule.
\begin{itemize}[leftmargin=2cm]
\item \texttt{batiment} : Ensembles d'étages et d'escaliers. Fonctions de calcul d'interaction entre humains et d'évolutions des coordonnées des humains.
\item \texttt{etage} : Ensemble de cellule. Fonctions de calcul de plus court chemin vers les sorties.
\item \texttt{escalier} : Liaison entre les étages. 
\item \texttt{cellule} : Contient des informations (direction vers la sortie, densité d'humain...) et un statut (vide, mur, sortie, entrée)
\item \texttt{foule} : Ensemble des humains.
\item \texttt{humain} : nouvelles coordonnées, ancienne et actuel. Forces appliquées. Caractère (nervosité)
\end{itemize}
%
\subsection{La vue}
Construit une représentation graphique du système et affiche celle-ci.
%
\subsection{Le controleur}
Exécute alternativement 
\begin{itemize}[leftmargin=2cm]
\item l'affichage de la vue
\item l'évolution du modèle
\item les actions du clavier.
\end{itemize}
%
%%%%%%%%%%%%%%%%%%%%%%%%%%%%%%%%%%%%%%%%%%%%%%%%%%%%%%%%%%%%%%%%%%%
