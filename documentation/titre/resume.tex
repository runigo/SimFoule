\begin{center}
\Large
Résumé
\normalsize
\end{center}
\vspace{3cm}
\begin{itemize}[leftmargin=1cm, label=\ding{32}, itemsep=21pt]
\item {\bf Objet : }Ce document (en cours de construction), accompagne le programme SimFoule (lui même en cours de développement).
\item {\bf Contenu : }Il contient un manuel d'installation et d'utilisation ainsi que quelques développements théoriques liés à ce programme de simulation.
\item {\bf Public concerné : }Ce document s'adresse aux enseignants et aux étudiants du supérieur des sections sciences physiques et informatique. Il s'adresse également à tout les passionnés d'informatique et de la langue de Molière.
\end{itemize}

\vspace{3cm}

SimFoule est un simulateur numériques de foule offrant une représentation graphique et une interaction dynamique avec les paramètres physiques. Destinés à un usage pédagogique, il permet de visualiser le comportement d'une foule lors d'une évacation. Cette documentation accompagne ce programme.

\begin{itemize}[leftmargin=1cm, label=\ding{32}, itemsep=11pt]
\item Le premier chapitre présente le simulateur, fourni une procédure d'installation et précise les commandes permettant l'interaction avec le programme.
\item Le chapitre suivant fourni un certain nombre de développements théoriques liés à la modélisation physique des foules.
\item Enfin, le dernier chapitre rassemble les informations liées à la structure du programme.
\end{itemize}
